\begin{abstract}
This paper extends the concepts of informative selection, population distribution and sample distribution to a spatial process context. These notions were first defined %by \cite{pfefferman_1992}
in a context where the output of the random process of interest consists of independent and identically distributed realisations for each individual of a population. %{\color{red} \cite{bonnery2012uniform}}
It has been showed that informative selection induces stochastic dependence among realisations on selected units. In the context of spatial processes, the ``population'' is a continuous space and realisations for two different elements of the population are not independent. %We show how informative selection may induce a different dependence among selected units, how the sample distribution differs from the "population" distribution, and how one can  account for this effect in an simulated study when doing statistical inference, including Semi-variogram parametric and semi parametric estimation, as well as prediction on the part of the space for which the random process was not observed.
We show how informative selection may induce a different dependence among selected units from a spatial process and how the sample distribution differs from the population distribution. %and how one can account for this effect in order to avoid biased estimates. %in an simulated study when doing statistical inference about the variogram.
%In particular, we provide a first solution on how to account for  informative selection in doing statistical inference about the variogram. %, and when a finite population is considered.

%We show how informative selection may induce a different dependence among selected units and how the sample distribution differs from the ``population'' distribution. We then focus on estimation of the variogram, and we provide a first solution to how account for the informative selection.
~\\
\textbf{Keywords}: analytic inference, point process, sample distribution, population distribution, sample likelihood, unequal probability sample, variogram.

\end{abstract}
